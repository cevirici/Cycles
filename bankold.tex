\dpp{1}{Suppose $ABCD$ is a parallelogram. Consider circles $w_1$ and $w_2$ such that $w_1$ is tangent to segments $AB$ and $AD$ and $w_2$ is tangent to segments $BC$ and $CD$. Suppose that there exists a circle which is tangent to lines $AD$ and $DC$ and externally tangent to $w_1$ and $w_2$. Prove that there exists a circle which is tangent to lines $AB$ and $BC$ and also externally tangent to circles $w_1$ and $w_2$.}
\dpp{2}{Consider $m+1$ horizontal and $n+1$ vertical lines ($m,n\ge 4$) in the plane forming an $m\times n$ table. Cosider a closed path on the segments of this table such that it does not intersect itself and also it passes through all $(m-1)(n-1)$ interior vertices (each vertex is an intersection point of two lines) and it doesn't pass through any of outer vertices. Suppose $A$ is the number of vertices such that the path passes through them straight forward,  $B$ number of the table squares that only their two opposite sides are used in the path, and $C$ number of the table squares that none of their sides is used in the path. Prove that
$$A=B-C+m+n-1.$$}
\dpp{3}{In a circle $\Gamma_{1}$, centered at $O$, $AB$ and $CD$ are two unequal in length chords intersecting at $E$ inside $\Gamma_{1}$. A circle $\Gamma_{2}$, centered at $I$ is tangent to $\Gamma_{1}$ internally at $F$, and also tangent to $AB$ at $G$ and $CD$ at $H$. A line $l$ through $O$ intersects $AB$ and $CD$ at $P$ and $Q$ respectively such that $EP = EQ$. The line $EF$ intersects $l$ at $M$. Prove that the line through $M$ parallel to $AB$ is tangent to $\Gamma_{1}$}
\dpp{4}{Let $O$ denote the circumcentre of an acute-angled triangle $ABC$. Let point $P$ on side $AB$ be such that $\angle BOP = \angle ABC$, and let point $Q$ on side $AC$ be such that $\angle COQ = \angle ACB$. Prove that the reflection of $BC$ in the line $PQ$ is tangent to the circumcircle of triangle $APQ$.}
\dpp{5}{Find all polynomial $P(x)\in \mathbb{R}[x]$ such that $P(x-P(x))=P(x)-P(P(x))$ for all $x\in \mathbb{R}$}
\dpp{6}{Let it \(k\) be a fixed positive integer. Alberto and Beralto play the following game:
Given an initial number \(N_0\) and starting with Alberto, they alternately do the following operation: change the number \(n\) for a number \(m\) such that \(m < n\) and \(m\) and \(n\) differ, in its base-2 representation, in exactly \(l\) consecutive digits for some \(l\) such that \(1 \leq l \leq k\).
If someone can't play, he loses.

We say a non-negative integer \(t\) is a winner number when the gamer who receives the number \(t\) has a winning strategy, that is, he can choose the next numbers in order to guarrantee his own victory, regardless the options of the other player.
Else, we call it loser.

Prove that, for every positive integer \(N\), the total of non-negative loser integers lesser than $2^N\) is \(2^{N-\lfloor \frac{log(min\{N,k\})}{log 2} \rfloor}\)}
\dpp{7}{Find all primes $p$ and $q$ such that $3p^{q-1}+1$ divides $11^p+17^p$.}
\dpp{8}{The sequence $a_n$ is defined as follows: $a_1=1$ and for any $n\in\mathbb N$, the number $a_{n+1}$ is obtained from $a_n$ by adding $3$ if $n$ is a member of this sequence, and $2$ otherwise. Show that $a_n<(1+\sqrt 2)n$ for all $n$.}
\dpp{9}{For which positive integers $k$ can we construct a sequence $a_0,a_1,a_2,\dots$, where for each $i$, we have $a_{i+1}=a_{i}+k$ or $a_{i+1}=a_{i}-k$ or $a_{i+1}=a_{i}\times k$ or $a_{i+1}=a_{i}/k$?}
\dpp{10}{Determine if the following statement is true: given any non-negative $\lambda_{i,j}$, $1\le i<j\le n$, there always exists nonnegative reals $a_i, 1\le i\le n$ such that $|a_i-a_j|\ge \lambda_{i,j}$ for all $1\le i<j\le n$, and $$\sum_{i=1}^n a_i\le \sum_{1\le i<j\le n}\lambda_{i,j}$$}